\documentclass[11pt,oneside, letterpaper]{article}
\usepackage{amsmath}
\usepackage[parfill]{parskip}

\DeclareMathOperator*{\argmin}{arg\,min}

\newcommand{\btheta}{\boldsymbol{\theta}}
\newcommand{\bbeta}{\boldsymbol{\beta}}
\newcommand{\balpha}{\boldsymbol{\alpha}}
\newcommand{\bdelta}{\boldsymbol{\delta}}
\newcommand{\bepsilon}{\boldsymbol{\epsilon}}
\newcommand{\bmu}{\boldsymbol{\mu}}
\newcommand{\bSigma}{\boldsymbol{\Sigma}}
\newcommand{\bsigma}{\boldsymbol{\sigma}}

\newcommand{\bXpX}{\textbf{X}^\prime\textbf{X}}
\newcommand{\bzero}{\textbf{0}}
\newcommand{\bX}{\textbf{X}}
\newcommand{\bY}{\textbf{Y}}
\newcommand{\bZ}{\textbf{Z}}
\newcommand{\ba}{\textbf{a}}
\newcommand{\bb}{\textbf{b}}
\newcommand{\bc}{\textbf{c}}
\newcommand{\by}{\textbf{y}}
\newcommand{\bV}{\textbf{V}}
\newcommand{\bW}{\textbf{W}}
\newcommand{\bx}{\textbf{x}}

\newcommand{\MSE}{\text{MSE}}
\newcommand{\Var}{\text{Var}}
\newcommand{\Cov}{\text{Cov}}
\newcommand{\Corr}{\text{Corr}}
\newcommand{\Exp}{\text{E}}


\title{Not Much of an Article}
\author{Oliver Schabenberger}
\date{August 5, 2024}   % You can use \today for the current date

\begin{document}
\maketitle

\newcommand{\bincoef}[2]{{#1 \choose #2}}
\newcommand{\binco}[2][n]{{#1 \choose #2}}

\section{Introduction}

The Beta distribution is a member of the two-parameter
\textbf{exponential family} of distributions. 
\LaTeX\ ignores spaces, but there are some rules; a blank line starts a new paragraph.

A random variable has a Beta distribution with parameters $\alpha$ and $\beta$,
denoted $Y \sim \text{Beta}(\alpha,\beta)$, if its density function is given 
by 
$$
f(y) = \frac{\Gamma(\alpha+\beta)}{\Gamma(\alpha)\Gamma(\beta)}\, y^{\alpha-1}\,(1-y)^{\beta-1}\quad 0 < y < 1
$$

\section{Basic Math}

\subsection{Subscripts and superscripts}

\begin{itemize}
	\item Subscripts: $x_1$, $x_{12}$
	\item Superscripts: $a^2$, $y^{(t)}$
	\item Operators with subscripts/superscripts:
	$$ \sum_{i=1}^n Y_{ij} \qquad  \int_{-\infty}^\infty f(x) dx$$
	$$ \prod_{k=1}^m p(y_k) \qquad \int_0^1 f(u) du$$
	
\end{itemize}

In inline mode, the formulas are typeset differently so that they fit within the line width,
for example $\sum_{i=1}^n (y_i - \overline{y})^2$.

Double subscripts/superscripts are possible, but often frowned upon. You have to be clever
with the syntax for \LaTeX\ to not throw an error:
$$
Y_{i_{jk}}  \qquad Y_{i_{j_k}}
$$

\subsection{Parentheses and brackets}

\begin{itemize}
	\item Parentheses: \verb|(x+y)| renders $(x+y)$
	\item Brackets: \verb|[x+y]| renders $[x+y]$
	\item Curly Braces: \verb|\{x + y\}| renders $\{x + y\}$
	\item Angle Brackets: \verb|\langle x + y \rangle| renders $\langle x + y \rangle$
	\item Vertical Bars: \texttt{|x+y|} renders $|x + y|$
\end{itemize}

You can size elements using \verb|\bigl|, \verb|\Bigl|, \verb|\biggl|, and \verb|Biggl|:

$$
\bigl\{ \quad \Bigl\{ \quad \biggl\{ \quad \text{ and } \Biggl\{
$$

It is easier to use \verb|\left| and \verb|\right| to automatically size parentheses to the height of the enclosed expression:

$$
\sum_{i=1}^n \left( \frac{Y_{ij} - \overline{Y}} {\left|Y_{ij}\right| - \text{median}(Y)} \right )^2
$$

\subsection{Spacing}

\begin{align*}
f(x,\!a) &= x^2\! +5x\! +a \\
f(x,a) &= x^2+5x+a \\
f(x,\,a) &= x^2\, +5x\, +a \\
f(x,\:) &= x^2\: +5x\: +a \\
f(x,\;a) &= x^2\; +5x\; +a \\
f(x,\ a) &= x^2\ +5x\ +a \\
f(x,\quad a) &= x^2\quad +5x\quad +a \\
f(x,\qquad a) &= x^2\qquad +5x\qquad +a
\end{align*}

\subsection{Arrays and matrices}

$$
\begin{bmatrix}
   a_{11} & a_{12} & a_{13} \\
   a_{21} & a_{22} & a_{23}
\end{bmatrix}
$$

$$
\left [
\begin{array}{ccc}
   a_{11} & a_{12} & a_{13} \\
   a_{21} & a_{22} & a_{23}
\end{array}
\right ]
$$

\section{Commands: Binomial Coefficients}

$$
\bincoef{n}{2}
$$

$$
\binco{2} \qquad \binco[6]{3}
$$

\section{Multi-line Equations}

Aligning equations with the \texttt{align} environment.

\begin{align*}
        \MSE (h(\bY);\theta) &= \text{E} \left [ \left( h(\bY) - \theta \right)^2\right] \\
        &= \Exp \left [h(\bY)^2 - 2 h(\bY) \theta + \theta^2\right ]\\
        &= \Exp \left [h(\bY)^2 - \mu^2 + \mu^2 - 2 h(\bY) \theta + \theta^2\right ]\\
        &= \Exp \left [h(\bY)^2 \right ] - \mu^2  + \mu^2 - 2 \mu \theta + \theta^2 \\
        &= \Exp \left [h(\bY)^2 \right ] - \mu^2  + (\mu - \theta)^2 \\
        &= \Var[h(\bY)] + \text{Bias}(h(\bY);\theta)^2
\end{align*}


If you do not care about alignment, then you can use the \verb|eqnarray| environment:

\begin{eqnarray*}
        \MSE (h(\bY);\theta) = \text{E} \left [ \left( h(\bY) - \theta \right)^2\right] \\
        = \Exp \left [h(\bY)^2 - 2 h(\bY) \theta + \theta^2\right ]\\
        = \Exp \left [h(\bY)^2 - \mu^2 + \mu^2 - 2 h(\bY) \theta + \theta^2\right ]\\
        = \Exp \left [h(\bY)^2 \right ] - \mu^2  + \mu^2 - 2 \mu \theta + \theta^2 \\
        = \Exp \left [h(\bY)^2 \right ] - \mu^2  + (\mu - \theta)^2 \\
        = \Var[h(\bY)] + \text{Bias}(h(\bY);\theta)^2
\end{eqnarray*}


\begin{align*}
Y &\sim \text{Bernoulli}(\mu) \\
\eta         &= \bx^\prime\bbeta \\
g(\mu)       &= \log \left(\frac{\mu}{1-\mu} \right) = \eta \\
g^{-1}(\eta) &= \frac{1}{1+e^{-\eta}} = \mu
\end{align*}

\LaTeX\ is not sensitive to spacing. The following code produces the same equation
but is much less readable.

\begin{align*} Y &\sim \text{Bernoulli}(\mu) \\ \eta &= \bx^\prime\bbeta \\
g(\mu) &= \log \left(\frac{\mu}{1-\mu} \right) = \eta \\
g^{-1}(\eta) &= \frac{1}{1+e^{-\eta}} = \mu \end{align*}

\section{Tables}

\begin{center}
\begin{tabular}{c | r r}
Observation & $X_1$ & $X_2$ \\
\hline
1 & 0.1 & -0.3 \\
2 & 0.2 & -0.4 \\
3 & 1.4 & 10.3 \\
\hline
\end{tabular}
\end{center}

\end{document}

